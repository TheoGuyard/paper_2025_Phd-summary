%!TEX aux_directory = aux
%!TEX output_directory = aux
\documentclass[11pt]{article}

% Style file options
%   - []: enable editions macros
%   - [final]: remove edition macros
\usepackage[]{TGstyle}

% ------ Metadata ------ %

\title{Thesis Manuscript Summary}
\author{Théo Guyard \\ \small{Insa and Inria Rennes}}
\date{\today}

% ------ Headings ------ %

% Encoding
\usepackage[utf8]{inputenc}
\usepackage[T1]{fontenc}

% Colors
\usepackage[dvipsnames]{xcolor}
\definecolor{mycolor}{RGB}{41,128,185}

% Glossaries
\usepackage{glossaries}
\glsdisablehyper

% Text
\usepackage[hidelinks]{hyperref}
\hypersetup{
  colorlinks=true,
  linkcolor=BrickRed,
  urlcolor=NavyBlue,
  citecolor=NavyBlue,
}
\usepackage{cleveref}
\usepackage{lipsum}
\usepackage{enumitem}
\usepackage{comment}

% Math formatting
\usepackage{amsmath}
\usepackage{stmaryrd}

% Graphics
\usepackage{booktabs}
\usepackage{tikz}
\usepackage{pgfplots}
\pgfplotsset{compat=newest}
\usepackage[ruled]{algorithm2e}

% Bibliography
\usepackage[numbers,sort]{natbib}
\bibliographystyle{abbrvnat}

% Theorem environments
\newtheorem{assumption}{Assumption}
\newtheorem{definition}{Definition}
\newtheorem{corollary}{Corollary}
\newtheorem{example}{Example}
\newtheorem{hypothesis}{Hypothesis}
\newtheorem{lemma}{Lemma}
\newtheorem{proposition}{Proposition}
\newtheorem{remark}{Remark}
\newtheorem{theorem}{Theorem}

% Crefnames
\crefname{assumption}{assumption}{assumptions}
\Crefname{assumption}{Assumption}{Assumptions}

% Problems
\newcommand{\ppb}{\mathcal{P}}

% Objective values and functions
\newcommand{\pobj}{p}

% Variables
\newcommand{\pvletter}{x}
\newcommand{\pv}{\mathbf{\pvletter}}
\newcommand{\pvi}[1]{\pvletter_{#1}}

% Sets
\newcommand{\pset}{\mathcal{X}}

% Problem data
\newcommand{\pdim}{n}
\newcommand{\lfunc}{f}

% Indices
\newcommand{\idxentry}{i}

% Math operators
\DeclareMathOperator{\argmin}{argmin}
\DeclareMathOperator{\argmax}{argmax}
\DeclareMathOperator{\dom}{dom}
\DeclareMathOperator{\interior}{int}

% Math misc
\newcommand{\1}{\mathbf{1}}
\newcommand{\0}{\mathbf{0}}
\newcommand{\abs}[1]{|#1|}
\newcommand{\biconj}[1]{#1^{**}}
\newcommand{\card}[1]{|#1|}
\newcommand{\conj}[1]{#1^{*}}
\newcommand{\grad}{\nabla}
\newcommand{\icvx}{\eta}
\newcommand{\intervint}[2]{[#1,#2]}
\newcommand{\norm}[2]{\|#1\|_#2}
\newcommand{\opt}[1]{#1^{\star}}
\newcommand{\separable}[2]{#1_{#2}}
\newcommand{\subdiff}{\partial}
\newcommand{\transp}[1]{#1^{\mathrm{T}}}

% Abbreviations
\newcommand{\ie}{\textit{i.e.}}
\newcommand{\eg}{\textit{e.g.}}
\newcommand{\etal}{\textit{et al.}}

\newacronym{bnb}{BnB}{Branch-and-Bound}
\newacronym{mip}{MIP}{Mixed-Integer Programming}
\usepackage{kmath}

% ------ Edition macros ------ %

\ProvideEditionMacros{TG}{blue}

% ------ Document ------ %

\begin{document} 

\maketitle
\vspace*{-30pt}

The manuscript focuses on $\ell_0$-regularized optimization problems expressed as
\begin{equation}
    \label{prob:prob}
    \tag{$\ppb$}
    \textstyle
    \opt{\pobj} = \min_{\pv \in \kR^{\pdim}} \lfunc(\pv) + \reg\norm{\pv}{0} + \pfunc(\pv)
\end{equation}
where $\kfuncdef{\lfunc}{\kR^{\pdim}}{\kR \cup \{+\infty\}}$ is a loss function, the notation $\norm{\cdot}{0}$ stands for the $\ell_0$-norm which counts the number of non-zeros in its input to promote sparse solutions, and $\kfuncdef{\pfunc}{\kR^{\pdim}}{\kR \cup \{+\infty\}}$ is a penalty function used to promote other properties than sparsity.
This family of problems is of paramount interest in various fields such as signal processing, machine learning, or statistics, among many others \citep{tillmann2021cardinality}.
For instance, problem \eqref{prob:prob} allows to construct sparse linear regressors in compressed sensing applications \citep{tropp2010computational} and its solutions correspond to maximum \textit{a posteriori} estimators of  Bernoulli-Gaussian signals \citep{soussen2011bernoulli}.

Problem \eqref{prob:prob} can be cast into the \gls{mip} formalism and tackled using generic solvers \citep{bourguignon2015exact}.
However, the latter often achieve poor numerical performance since they are not able to exploit the sparse structure of the problem.
Hence, recent work directions have rather focused on designing specialized \gls{bnb} algorithms to achieve more reasonable solving times \citep{ben2022global,hazimeh2022sparse}.
From \Cref{sec:bnb,sec:contributions,sec:numerics}, we describe the main ingredients of \gls{bnb} algorithms tailored to problem \eqref{prob:prob}, outline the contributions presented in the manuscript for these methods, and give a glimpse of the implication of our work for signal processing applications.



\section{Branch-and-Bound Algorithms}
\label{sec:bnb}

A \gls{bnb} algorithm is a general paradigm to address global optimization problems.
It explores different \emph{regions} in the feasible space and \emph{prunes} those that cannot contain global solutions.
When applied to problem \eqref{prob:prob}, the \gls{bnb} algorithm starts from the whole feasible space $\kR^{\pdim}$ and create regions by driving the sparsity of the optimization variable, that is by fixing some entries to zero or non-zero.
Therefore, regions explored are defined as
\begin{equation}
    \label{eq:region}
    \nodesymbol = \left\{
        \pv \in \kR^{\pdim}
        \left|
        \begin{array}{ll}
            \pvi{\idxentry} = 0 &\forall \idxentry \in \setzero \\
            \pvi{\idxentry} \neq 0 &\forall \idxentry \in \setone \\
            \pvi{\idxentry} \in \kR &\forall \idxentry \in \setnone
        \end{array}
        \right.
    \right\}
\end{equation}
where $(\setzero,\setone,\setnone)$ corresponds to a partition of $\{1,\dots,\pdim\}$.
Tailored rules can be implemented to select index to fix and the next region to process during the \gls{bnb} algorithm \citep{samain2023techniques}.

\paragraph{Region processing.}
The central mechanism of the \gls{bnb} algorithm is to detect when a given region $\nodesymbol$ explored cannot contain solutions to problem \eqref{prob:prob} via a \emph{pruning test}.
Letting 
\begin{equation}
    \label{prob:prob-node}
    \tag{$\node{\ppb}$}
    \textstyle
    \node{\pobj} = \min_{\pv \in \nodesymbol} \lfunc(\pv) + \reg\norm{\pv}{0} + \pfunc(\pv)
    ,
\end{equation}
one can observe that no solutions can be contained in $\nodesymbol$ when $\node{\pobj} > \opt{\pobj}$ and the region can then be discarded from the problem's feasible space.
Since, evaluating this inequality is a complicated task in practice, one rather implements a pruning test expressed as
\begin{equation}
    \label{eq:pruning-test}
    \node{\LB{\pobj}} > \opt{\UB{\pobj}}
\end{equation}
for some bounds $\node{\LB{\pobj}} \leq \node{\pobj}$ and $\opt{\UB{\pobj}} \geq \opt{\pobj}$.
Implementing the \gls{bnb} algorithm then mainly amounts to devising strategies to compute these bounds, typically sought tight and tractable to ensure the efficiency of the pruning test.

\paragraph{Bounds evaluation.}
Various heuristic-based strategies allow to obtain some upper bound $\opt{\UB{\pobj}}$ on $\opt{\pobj}$ of good quality at a reasonable cost.
The real challenge lies in the computation of a lower bound $\node{\LB{\pobj}}$ for each region $\nodesymbol$ explored during the \gls{bnb} algorithm.
This task is usually achieved by solving a \emph{relaxation} of problem \eqref{prob:prob-node} expressed as
\begin{equation}
    \label{prob:relax-node}
    \tag{$\node{\rpb}$}
    \textstyle
    \node{\robj} = \min_{\pv \in \kR^{\pdim}} \lfunc(\pv) + \node{\LB{\rfunc}}(\pv)
\end{equation}
for some $\kfuncdef{\node{\LB{\rfunc}}}{\kR^{\pdim}}{\kR \cup \{+\infty\}}$.
The latter function must verify $\node{\LB{\rfunc}}(\pv) \leq \reg\norm{\pv}{0} + \pfunc(\pv)$ for all $\pv \in \nodesymbol$ to ensure that setting $\node{\LB{\pobj}} = \node{\robj}$ leads to a valid pruning test \eqref{eq:pruning-test}.
Moreover, one usually wants to construct $\node{\LB{\rfunc}}$ \emph{convex} to obtain a tractable relaxation \eqref{prob:relax-node}.
Several strategies have been proposed in the literature to design $\node{\LB{\rfunc}}$ depending on the expression of $\pfunc$ \citep{ben2022global,hazimeh2022sparse}.

\paragraph{Bottlenecks.}
\gls{bnb} algorithms proposed in the literature suffer from two main issues.
\begin{enumerate}[label=\arabic*., nosep]
    \item First, they are designed for a specific choice of function $\pfunc$ in the objective of problem \eqref{prob:prob}, thereby restricting their scope of application. This limitation is mainly due to the lack of generic methodology to construct the function $\node{\LB{\rfunc}}$ in the relaxation \eqref{prob:relax-node}, independently of the expression of $\pfunc$.
    \item Second, they can still suffer from poor numerical performance on some real-world instances of problem \eqref{prob:prob}. Specially, their solving time can be expressed as 
    \begin{equation}
        \label{eq:complexity}
        \textstyle
        T = \sum_{\nodesymbol \in N} t_{\nodesymbol}
    \end{equation}
    where $N$ corresponds to the set of all regions processed during the \gls{bnb} algorithm and $t_{\nodesymbol}$ corresponds to the time needed to process a given region $\nodesymbol \in N$. In practice, the cardinality of $N$ can be large and since processing each region requires solving a relaxation \eqref{prob:relax-node}, this can result in prohibitive solving times.
\end{enumerate}
The manuscript proposes theoretical and practical contributions to these challenges.


\section{Contributions}
\label{sec:contributions}

In the following, we outline the main contributions of the manuscript.

\subsection{Generalizing the \gls{bnb} Algorithm}
\label{sec:contributions:generic}

The first contribution introduces a generic strategy for constructing the relaxation \eqref{prob:relax-node} within the \gls{bnb} algorithm.
Our approach only necessitates mild assumptions and does not depend on a specific expression of the function $\pfunc$.
It mainly stems from the following result.
\begin{theorem}
    \label{thm:biconj}
    Let $\ggfunc(\pvi{}) = \reg\norm{\pvi{}}{0} + \gpfunc(\pvi{})$ where $\pvi{} \in \kR$, $\reg > 0$ and $\kfuncdef{\gpfunc}{\kR}{\kR \cup \{+\infty\}}$ is a proper, closed, convex, even, and supercoercive function verifying $\gpfunc \geq \gpfunc(0) = 0$ with $0 \in \interior\dom(\gpfunc)$.
    Then, the convex envelope of $\ggfunc$ is given by
    \begin{equation}
        \biconj{\ggfunc}(\pvi{}) = 
        \begin{cases}
            \pslope \abs{\pvi{}} &\text{if } \abs{\pvi{}} \leq \plimit \\
            \gpfunc(\pvi{}) + \reg &\text{otherwise}
        \end{cases}
    \end{equation}
    with $\pslope = \sup\kset{\bvi{} \in \kR}{\conj{\gpfunc}(\bvi{}) \leq \reg}$, $\plimit = \sup\kset{\bvi{} \in \kR}{\bvi{} \in \subdiff\conj{\gpfunc}(\pslope)}$, and where $\conj{\gpfunc}$ denotes the convex conjugate of $\gpfunc$.
\end{theorem}
As a byproduct of \Cref{thm:biconj}, we provide a principled manner for constructing the function $\node{\LB{\rfunc}}$ in the relaxation \eqref{prob:relax-node}, independent of the expression of $\pfunc$. Notably, this strategy encompasses prior results from the literature and generalizes them to a broader framework.
Beyond this generic method for constructing relaxations, we also characterize various operators associated with the function $\node{\LB{\rfunc}}$ and essential for implementing solution methods tailored to problem \eqref{prob:relax-node}.
Overall, this contribution gives a generic backbone for a \gls{bnb} algorithm capable of addressing a wide range of instances of problem \eqref{prob:prob}.

\subsection{Accelerating the Relaxation Processing}

Our second contribution aims at reducing the processing time $t_{\nodesymbol}$ per region in \eqref{eq:complexity} to improve the \gls{bnb} algorithm efficiency.
To do so, we propose an efficient numerical procedure to address the relaxations \eqref{prob:relax-node} constructed during the \gls{bnb} algorithms which correspond to convex optimization problems.
In particular, we have adapted a coordinate-descent method to tackle problem \eqref{prob:relax-node}, which turns out to be a particularly efficient choice in our context.
This method is also enhanced via acceleration strategies based on working-sets and screening tests to further reduce the relaxation processing time.
The resulting numerical procedure allows to address each relaxation constructed during the \gls{bnb} algorithms efficiently


\subsection{Reducing the Number of Regions Processed}

Our third contribution aims at reducing the number $N$ of regions explored in \eqref{eq:complexity} to improve the \gls{bnb} algorithm efficiency.
To do so, we first show that 
\begin{equation}
    \label{eq:dual-bound}
    \node{\pobj} \geq \node{\dfunc}(\dv)
\end{equation}
for all $\dv \in \kR^{\pdim}$, where $\node{\dfunc}$ corresponds to the objective function of the Fenchel-Rockafellar problem associated to \eqref{prob:relax-node} for which we provide a closed-form expression.
Inequality \eqref{eq:dual-bound} allows to implement the pruning test \eqref{eq:pruning-test} with $\node{\LB{\pobj}} = \node{\dfunc}(\dv)$.
Such \emph{dual} lower-bound have already been exploited in the literature to \AddTodo{XXX}.
We have proposed a novel strategy to further exploit these bounds to reduce the 


With such bounds, one need not solve a relaxation \eqref{prob:relax-node} but only to evaluate a function to assess whether a region can be pruned.
However, the quality of the resulting pruning test is potentially degraded.



that can be implemented in the pruning test \eqref{eq:pruning-test}.

\begin{itemize}
    \item Dual bounds
    
    where stems from Fenchel-Rockafellar.
    \begin{theorem}
        \label{thm:simpruning}
        \begin{subequations}
            \begin{align}
                \dfunc^{\childzero{\nodesymbol}{\idxentry}}(\dv) &= \node{\dfunc}(\dv) + \pivotzero{\idxentry}(\dv) \\
                \dfunc^{\childone{\nodesymbol}{\idxentry}}(\dv) &= \node{\dfunc}(\dv) + \pivotone{\idxentry}(\dv)
            \end{align}
        \end{subequations}
        where \AddTodo{blabla}
    \end{theorem}
    \item The relation highlighted in \Cref{thm:simpruning} can be exploited to reduce the number of regions processed during the \gls{bnb} algorithm at a low computational cost. While processing a given region $\nodesymbol$, one can evaluate a pruning test \eqref{eq:pruning-test} for a nested region $\nodesymbol' \subseteq \nodesymbol$ that has not been explored using a dual bound. The latter can be efficiently computed from the relation \eqref{eq:dual-bound}.
    \item This is particularily efficient because \AddTodo{blablabla}
\end{itemize}


\subsection{A Practical Solver}

Our contributions are implemented in \texttt{el0ps}, a Python package available online at
\begin{center}
\url{https://github.com/TheoGuyard/El0ps}
\end{center}
which provides a framework for constructing and solving instances of problem \eqref{prob:prob}.
It includes several build-in instances and offers users the flexibility to create custom ones by specifying the expressions for the functions $\lfunc$ and $\pfunc$ of interest for their applications, along with some associated operators (sub-differential, proximal operator, etc...).
Once a problem instance has been defined, it can be automatically processed using a \gls{bnb} algorithm.
Experimental results demonstrate that our solver achieves state-of-the-art performance, with acceleration factors ranging from one to several orders of magnitude in terms of solving times compared to its best competitors.


\section{Numerical illustrations}
\label{sec:numerics}


\bibliography{main}

\clearpage

\thispagestyle{empty}

\vspace*{150pt}
\begin{center}
    \LARGE\bf{Lettre de Candidature}
\end{center}
\vspace*{50pt}
\begin{flushright}
    À Rennes, le 16 décembre 2024
\end{flushright}
\vspace*{20pt}
\textbf{Objet:} Candidature au prix de thèse 2025 du GRETSI

\vspace*{20pt}

Par cette présente lettre, je souhaiterais soumettre ma candidature au prix de thèse 2025 du GRETSI pour mon manuscrit intitulé \textit{Branch-and-Bound Algorithms for L0-Regularized Optimization Problems} réalisé sous la direction de M. Cédric Herzet.
Je certifie que les critères de recevabilité de ma candidature sont respectés et m'engage à fournir tous les documents nécessaires à l'examen de mon dossier.
~\\

Veuillez agréer, Madame, Monsieur, l'expression de mes salutations distinguées.

\vspace*{50pt}
\begin{flushright}
    Théo Guyard
\end{flushright}

\end{document}